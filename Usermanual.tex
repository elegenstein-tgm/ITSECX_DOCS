% !TEX TS-program = pdflatex
% !TEX encoding = UTF-8 Unicode

% This is a simple template for a LaTeX document using the "article" class.
% See "book", "report", "letter" for other types of document.

\documentclass[11pt]{article} % use larger type; default would be 10pt

\usepackage[utf8]{inputenc} % set input encoding (not needed with XeLaTeX)

%%% Examples of Article customizations
% These packages are optional, depending whether you want the features they provide.
% See the LaTeX Companion or other references for full information.

%%% PAGE DIMENSIONS
\usepackage{geometry} % to change the page dimensions
\geometry{a4paper} % or letterpaper (US) or a5paper or....
% \geometry{margin=2in} % for example, change the margins to 2 inches all round
% \geometry{landscape} % set up the page for landscape
%   read geometry.pdf for detailed page layout information

\usepackage{graphicx} % support the \includegraphics command and options

% \usepackage[parfill]{parskip} % Activate to begin paragraphs with an empty line rather than an indent

%%% PACKAGES
\usepackage{booktabs} % for much better looking tables
\usepackage{array} % for better arrays (eg matrices) in maths
\usepackage{paralist} % very flexible & customisable lists (eg. enumerate/itemize, etc.)
\usepackage{verbatim} % adds environment for commenting out blocks of text & for better verbatim
\usepackage{subfig} % make it possible to include more than one captioned figure/table in a single float
\usepackage{ulem}  % Strikeout with sout
% These packages are all incorporated in the memoir class to one degree or another...

%%% HEADERS & FOOTERS
\usepackage{fancyhdr} % This should be set AFTER setting up the page geometry
\pagestyle{fancy} % options: empty , plain , fancy
\renewcommand{\headrulewidth}{0pt} % customise the layout...
\lhead{}\chead{}\rhead{}
\lfoot{}\cfoot{\thepage}\rfoot{}

%%% SECTION TITLE APPEARANCE
\usepackage{sectsty}
\allsectionsfont{\sffamily\mdseries\upshape} % (See the fntguide.pdf for font help)
% (This matches ConTeXt defaults)

%%% ToC (table of contents) APPEARANCE
\usepackage[nottoc,notlof,notlot]{tocbibind} % Put the bibliography in the ToC
\usepackage[titles,subfigure]{tocloft} % Alter the style of the Table of Contents
\renewcommand{\cftsecfont}{\rmfamily\mdseries\upshape}
\renewcommand{\cftsecpagefont}{\rmfamily\mdseries\upshape} % No bold!

%%% END Article customizations

%%% The "real" document content comes below...

\title{Usermanual for ITSecX}
\author{Erik Brändli}
%\date{18/05/2015} % Activate to display a given date or no date (if empty),
         % otherwise the current date is printed 

\begin{document}
\maketitle

\section{Einleitung}

EditierenITSecX ist die Abkürzung für "IT - Security - Extreme". ITSecX lässt sich mit dem von uns zur Verfügung gestellten Devices verbinden.\\
Die interne Syntax ist C \# . Das Hauptziel des Projektes ist es zu zeigen, wie einfach man Daten Abfangen und auswerten kann und Amateur Pen-Testern die Möglichkeit zu geben, schnell dynamisch an Ziele gelangen. Es ist auch möglich aus der zur Verfügung gestellten Software weit mehr rauszuholen (gegen Aufpreis).\\
Dieses Handbuch besteht vorranging aus einer Funktionsreferenz, Erläuterungen zu den wichtigsten Features und weitere ergänzende Informationen.

%Sie können dieses Handbuch in verschiedenen Formaten unter » http://www.php.net/download-docs.php herunterladen. Informationen dazu, wie dieses Handbuch erstellt wird, finden Sie %im Anhang unter dem Kapitel 'Über dieses Handbuch'. Wenn Sie sich für die Geschichte von PHP interessieren, lesen Sie bitte den entsprechenden Anhang.

\subsection{Autoren und Mitwirkende}

Das Projektteam bestand aus 5 Mitgliedern von denen aktuell noch 2 vorhanden sind. 

-) Erik Brändli 

-) Hüseyin Bozkurt

\sout{-) Markus Schulmeister}

\sout{-) Arian Sayah} 

\sout{-) Raied El'beidak}
\\ \\
Projektabnehmer:

-) Michael Borko

-) Werner Kristufek

-) Erich Trenner

-) Elisabeth Wildling

\subsection{Autoren und Editoren}

Folgende Personen verdienenen Anerkennung dafür, dass Sie wesentlichen Inhalt zum Handbuch beigetragen haben und/oder weiterhin beitragen werden: Hüsyein Bozkurt und Erik Brändli.

Vielen Dank!

\subsection{Einsatzgebiet}

Als Einsatzgebiet sind Schulen mit IT-Ausbildung gedacht, da man dort den Schülern lehren, kann wie einfach es ist ein Netzwerk auszutricksen bzw. um zu veranschaulichen, dass Netzwerksicherheit ein wichtiger Part in unserer Welt ist.

Denn die meisten Schüler denken, dass ihre Daten sicher sind, und dass niemand mitlesen kann!

\section{Vorbereitung des Produktes}
Sollten Sie bereits unser modifiziertes "Arch Linux" besitzen können Sie dieses Kapitel überspringen.
\subsection{Vorausgesetztes}
Um das Produkt zu verwenden erwerben Sie am besten unser "Arch Linux" von unserer Seite, oder Sie insatllieren sich zu Ihrer Arch Linux version folgende Pakete dazu:

-) Openssh

-) openbsd-netcat

-) nmap

-) tcpdump

-) dsiff

\subsection{Erstellen eines tcpdump-services}
Falls Sie nicht unser modifiziertes "Arch Linux" verwenden müssen Sie sich einen Service schreiben welches einen tcpdump startet auf einem von Ihnen ausgewählten Interface startet. Dieser Dump muss in eine Datei geschrieben werden, wie dies funktioniert finden sie auf der Seite von tcpdump.(http://www.tcpdump.org/)\\

\subsection{Erweiterungen}
Es sollte sich der Tcpdump bei Systemstart starten

\end{document}
